\documentclass{article}

\usepackage{blindtext}
\usepackage[utf8]{inputenc}
\usepackage{fancyhdr}
\usepackage{amsmath}
\usepackage{amssymb}
\usepackage{amsthm}
\usepackage{tikz}
\usepackage{blkarray}
\usepackage{algorithm}
\usepackage{algpseudocode}
\usepackage{algorithmicx}
\usepackage{hyperref}
\usepackage{tikz}
\usepackage{pgfgantt}
\usepackage{listings}
\usepackage{xcolor}
\usepackage{wrapfig}
\usepackage{relsize}
\usepackage{eurosym}
\usepackage{parskip}
\usepackage{textcomp}
\usepackage{stackengine}
\usepackage{float}
\usepackage{bookmark}
\usepackage{caption}
\usepackage{forest}
\usepackage{mdframed}
\usepackage{cancel}
\usepackage{enumitem}
\usepackage{tabu}
\usepackage{tcolorbox}
\usepackage{multicol}
\tcbuselibrary{theorems,breakable}
\usepackage[a4paper, total={6in, 8in}]{geometry}
\usetikzlibrary{calc,trees,positioning,arrows,chains,shapes.geometric,decorations.pathreplacing,decorations.pathmorphing,shapes,matrix,shapes.symbols,fit,graphs,snakes}
\usepackage{nicefrac}


% Migliore grafica per i theorem/definition/properties/exercises/examples
\newtheoremstyle{break}% name
{2 \baselineskip}%         Space above, empty = `usual value'
{}%         Space below
{} %{\itshape}% Body font. I prefer empty body font
{\parindent}%         Indent amount (empty = no indent, \parindent = para indent)
{\bfseries}% Thm head font
{.}%        Punctuation after thm head
{\newline}% Space after thm head: \newline = linebreak
{}%         Thm head spec

\theoremstyle{break}

\newtcbtheorem[number within=section]{thm}{Teorema}%
{colback=green!5,colframe=green!35!black,fonttitle=\bfseries}{th}
\newcounter{def}
\newtcbtheorem[number within=section]{definition}{Definizione}%
{colback=red!5,colframe=red!35!black,fonttitle=\bfseries}{dfn}
\newcounter{prop}
\newtheorem{property}[prop]{Proprietà}
\newcounter{exer}
\newtheorem{exercise}[exer]{Esercizio}
\newcounter{exmp}
\newtcbtheorem[number within=section]{example}{Esempio}%
{breakable,colback=gray!5,colframe=gray!35!black,fonttitle=\bfseries}{exmp}
\newcounter{lmma}
\newtheorem{lemma}[lmma]{Lemma}
\tcolorboxenvironment{nota}{
blanker,breakable,left=5mm,
before skip=10pt,after skip=10pt,
borderline west={1mm}{0pt}{red}}

% TikzStyles
\tikzstyle{block} = [rectangle, draw, text centered]
\tikzstyle{empty_block} = [rectangle, text centered]
\tikzstyle{line} = [draw, -latex']
% /End TikzStyles

\definecolor{azzurro}{cmyk}{1,0.33,0,0.13}
\definecolor{arancione}{cmyk}{0,0.41,1,0}
\definecolor{verde}{cmyk}{0.44,0,0.38,0.31}
\definecolor{viola}{rgb}{0.58,0,0.82}
\definecolor{bianco}{rgb}{0.95, 0.95, 0.92}
% C style
\lstdefinestyle{CStyle}{
    backgroundcolor=\color{bianco},   
    commentstyle=\color{verde},
    keywordstyle=\color{viola},
    numberstyle=\tiny\color{arancione},
    stringstyle=\color{azzurro},
    basicstyle=\footnotesize,
    breakatwhitespace=false,         
    breaklines=true,                 
    captionpos=b,                    
    keepspaces=true,                 
    numbers=left,                    
    numbersep=5pt,                  
    showspaces=false,                
    showstringspaces=false,
    showtabs=false,                  
    tabsize=2,
    language=C
}
\lstset{
  basicstyle=\ttfamily,
  columns=fullflexible,
  frame=single,
  breaklines=true,
  postbreak=\mbox{\textcolor{red}{\(\hookrightarrow\)}\space},
}
% maximum matrix environment set to 50 instead of 10
\setcounter{MaxMatrixCols}{50}

\newcommand{\norm}[1]{\left\lVert#1\right\rVert}
\newcommand{\abs}[1]{\left|#1\right|}
\newcommand{\R}{\mathbb{R}}
\newcommand*\circled[1]{\tikz[baseline= (char.base)]{
            \node[shape=circle,draw,inner sep=2pt] (char) {#1};}}
\newcommand{\coloredbox}[1]{\fcolorbox{black}{#1}{\rule{0pt}{6pt}\rule{6pt}{0pt}}\quad}

% Text under-above matrix https://tex.stackexchange.com/a/78467
\newenvironment{spmatrix}[1]
  {\def\mysubscript{#1}\mathop\bgroup\begin{pmatrix}}
  {\end{pmatrix}\egroup_{\textstyle\mathstrut\mysubscript}}

\newenvironment{sbmatrix}[1]
  {\def\mysubscript{#1}\mathop\bgroup\begin{bmatrix}}
  {\end{bmatrix}\egroup_{\textstyle\mathstrut\mysubscript}}

% per le nostre cagate
\newcommand{\checkeditem}{\item[\refstepcounter{enumi}$\text{\rlap{$\checkmark$}}\square$]}
\newlist{todolist}{itemize}{2}
\setlist[todolist]{label=$\square$}

% Euro workaround - https://tex.stackexchange.com/questions/110972/eurosym-seems-to-not-be-working
\DeclareRobustCommand{\officialeuro}{%
  \ifmmode\expandafter\text\fi
  {\fontencoding{U}\fontfamily{eurosym}\selectfont e}}

% forest
\forestset{
  every leaf node/.style={
    if n children=0{#1}{}
  },
  every tree node/.style={
    if n children=0{}{#1}
  },
  edgelabel/.style n args={2}{
    edge label={node[midway,#1]{#2}}
  },
  nodevalue/.style n args={2}{
    label=#1:{{#2}}
  }
}
\forestset{
  suffix tree/.style={
    for tree={
      edge={->},
      every tree node={
        circle, draw, minimum size=1.5em, s sep=1cm,
        s sep+=1em,
        l sep+=1em
      },
      every leaf node={
        rectangle, draw, 
        minimum size=1.5em
      }
    }
  }
}
\forestset{
  highlight ancestors/.style={
    before typesetting nodes={
      for current and ancestors=highlight
    }
  },
  highlight/.style={
    draw=darkgray,
    edge+=red
  }
}
\forestset{
  terminal/.style={
    if n children=0{
    tier=terminal
    }{}
  }
}

\begin{document}
\section{I compitino, 18 novembre 2016 B}
Si richiede un algoritmo che sfrutta la tecnica della programmazione dinamica per risolvere il seguente problema: si considerino due sequenze $s_1$ e $s_2$ di lettere dell'alfabeto italiano, di lunghezza $n$ e $m$ rispettivamente. 

A ogni lettera dell'alfabeto è associato un numero naturale, mediante una funzione $f$ prefissata.

Si vuole calcolare la lunghezza di una più lunga sottosequenza comune a $s_1$ e $s_2$ nella quale i caratteri appaiono in ordine lessicografico crescente e i numeri ad essi associati appaiono in ordine decrescente.

\subsection{Variabili che servono per risolvere il problema}
La programmazione dinamica serve per risolvere un problema considerando di aver già risolto tutti i sottoproblemi (prefissi), e potendo usare la soluzione di essi per calcolare la soluzione del problema.

Per immagazzinare le soluzioni dei sottoproblemi viene utilizzata una matrice $c$, di dimensione $n \times m$. A ogni sottoproblema $i, j$ corrisponde una casella $c[i, j]$ che contiene il valore di una delle sottosequenze comuni più lunghe che rispettano i vincoli dell'algoritmo.

Con $s_1$ di lunghezza $n$ e $s_2$ di lunghezza $m$, il problema finale avrà dimensione $n,\ m$. Si definiscono indici $i, j$ che indicano ogni sottoproblema, tale che $0 \leq i \leq n$ e $0 \leq j \leq m$. In totale, considerando anche i casi limite con prefisso che corrisponde alla stringa vuota, si hanno $(n + 1)(m + 1)$ sottoproblemi.

Le sottostringhe $s_{1, i}$ e $s_{2, j}$ sono prefissi di lunghezza rispettivamente $i$ e $j$ delle stringhe di partenza $s_1$ e $s_2$ di lunghezza $n$ e $m$, con $0 \leq i \leq n$ e $0 \leq j \leq m$.

In questo caso, il problema è definito come: \\
\textit{trovare la lunghezza di una delle più lunghe sottosequenze di $s_1$, $s_2$ tale che i caratteri siano in ordine lessicografico crescente e i numeri associati in ordine decrescente}.

Ogni sottoproblema è definito come: \\
\textit{trovare la lunghezza di una delle più lunghe sottosequenze di $s_{1, i}$, $s_{2, j}$, con $0 \leq i \leq n$ e $0 \leq j \leq m$, tale che i caratteri siano in ordine lessicografico crescente e i numeri associati in ordine decrescente}.
	
Per risolvere questo problema, nel caso in cui $s_{1, i} = s_{2, j}$ è necessario conoscere il valore numerico dell'ultimo carattere della più lunga sottosequenza comune precedente, in modo da poterlo confrontare con il valore corrente.

Il problema viene esteso: \\
\textit{trovare la lunghezza di una delle più lunghe sottosequenze di $s_{1, i}$, $s_{2, j}$ tale che i caratteri siano in ordine lessicografico crescente e i numeri associati in ordine decrescente \textbf{e che termini con $s_{1, i}$, $s_{2, j}$} (nel caso in cui esse coincidano)}.

La variabile associata a ogni problema è: \\
\textit{$c[i, j]$ che contiene la lunghezza di una delle più lunghe sottosequenze di $s_{1, i}$, $s_{2, j}$ che abbia lettere crescenti e numeri associati decrescenti (nel caso in cui $s_{1, i}, s_{2, j}$ coincidano), e contiene 0 altrimenti}.

Si ricorda che ognuna di queste variabili è considerabile come una \textbf{black-box}: si può utilizzare ma non è possibile conoscerne il contenuto (si assume di aver risolto i sottoproblemi di dimensione minore).
	
\subsection{Equazione di ricorrenza}
Un algoritmo ricorsivo non è efficace, perché calcolerebbe più volte lo stesso risultato, quindi si ricorre alla programmazione dinamica per risolvere il problema.

Le soluzioni dei sottoproblemi non sono ancora note, ma è possibile utilizzarle per calcolare le soluzioni successive. L'unico caso conosciuto è il caso limite: se una delle due sequenze è vuota, la lunghezza della più lunga sottosequenza è 0 ($i = 0 \lor j = 0$).

Questo si può esprimere nel seguente modo:
$$c[i, 0] = 0\ con\ \{0 \leq i \leq n\},\ c[0, j] = 0\ con\ \{0 \leq j \leq m\}$$

Si introduce il sottoproblema tale per cui $s_{1, i} \neq s_{2, j}$: in questo caso, non è possibile trovare la più lunga sottosequenza comune che termini con $s_{1, i}, s_{2, j}$ con valore maggiore di 0. Di conseguenza si assegna 0 alla casella corrispondente.
$$c[i, j] = 0 \text{ con } s_{1, i} \neq s_{2, j}$$

Avendo risolto il caso con il prefisso vuoto, i sottoproblemi da risolvere diventano al più $mn$. L'equazione di ricorrenza per un generico sottoproblema $i, j$ è:
$$c[i, j] = \begin{cases}
1 + max\{c[h, k]\}\ \text{con}\ \{1 \leq h < i,\ 1 \leq k < j\} & \text{ se } s_{1, h} < s_{1, i},\ f(s_{1, i}) < f(s_{1, h}) \\
1 & \text{ altrimenti}  
\end{cases}$$
	
\subsection{Soluzione del problema}
La più lunga sottosequenza comune crescente con numeri decrescenti tra $s_{1, i}$ e $s_{2, j}$ è data dal massimo valore tra le soluzioni delle più lunghe sottosequenze comuni per i sottoproblemi di dimensione minori.

La soluzione del problema corrisponde al massimo tra tutte le caselle della matrice.
$$max\{c[i, j]\ |\ 0 \leq i \leq n \land 0 \leq j \leq m\}$$

L'algoritmo funziona in questo modo: in ogni casella di $c$ viene inserito 0 se i due caratteri in posizione $i, j$ sono diversi, altrimenti viene cercato il massimo nella sottomatrice precedente utilizzando due indici ausiliari $h, k$. Il valore corrispondente deve rispettare le seguenti condizioni: il numero precedente dev'essere maggiore, e la lettera precedente dev'essere minore della lettera associata.

Il risultato ottenuto viene incrementato di 1, perché è stato trovato un altro carattere uguale.

\subsection{Algoritmo in pseudocodice}
Viene illustrata la soluzione bottom-up mediante un algoritmo iterativo, risolvendo ogni sottoproblema una volta sola.
\begin{algorithm}[H]
	\caption{LGCS con numeri decrescenti}
	\begin{algorithmic}
		\Function{LGCS numeri decrescenti}{$s_1,\ s_2$}
			\For {$i \gets 0 \textbf{ to } n$}
				\State $c[i, 0] \gets 0$
			\EndFor
			\For {$j \gets 0 \textbf{ to } m$}
				\State $c[0, j] \gets 0$
			\EndFor
			\State $max \gets 0$
			\For {$i \gets 1 \textbf{ to } n$}
				\For {$i \gets 1 \textbf{ to } n$}
					\If {$s_1[i] \neq s_2[j]$}
						\State $c[i, j] \gets 0$
					\Else
						\State $temp \gets 0$
						\For {$h \gets 1 \textbf{ to } i-1$}
							\For {$k \gets 1 \textbf{ to } j-1$}
								\If {$s_{1, i} > s_{1, h} \land f(s_{1, i}) < f(s_{1, h}) \land c[h, k] > temp$}
									\State $temp \gets c[h, k]$
								\EndIf
							\EndFor
						\EndFor
						\State $c[i, j] \gets 1 + temp$
					\EndIf
					\If {$c[i, j] > max$}
						\State $max = c[i, j]$
					\EndIf
				\EndFor
			\EndFor
		\EndFunction
	\end{algorithmic}
\end{algorithm}

\subsection{Valutazione del tempo di esecuzione}
Il tempo di esecuzione di questo algoritmo è approssimativamente $O(n^2m^2)$. Ciò deriva dai cicli \textit{for} innestati: la matrice viene interamente analizzata casella per casella, e poi c'è la ricerca del massimo valore nella sottomatrice.

Lo spazio utilizzato è $\Theta(nm)$, per la matrice.

\end{document}
