\section{Algoritmo di Floyd-Warshall}
Dato un grafo pesato $G = (V, E, w)$, l'algoritmo determina il \textit{cammino minimo tra due nodi}. Esiste anche l'alternativa per ogni vertice a partire da una data sorgente.

Un algoritmo greedy in questo caso non funziona: il numero di possibilità è troppo elevato, e il cammino con il minor numero di archi potrebbe non essere il meno pesante. Un'alternativa è il calcolo della distanza senza vertici intermedi, e il miglioramento dei valori conoscendo tutte le distanze.

Se due nodi non sono direttamente collegati da un arco, inizialmente la distanza è infinita; in seguito, l'algoritmo prova a trovare percorsi con il minimo peso attraversando altri nodi.

Il calcolo iniziale delle distanze viene effettuato usando una matrice quadrata di dimensione $n \times n$, che indica quanto pesa il cammino minimo che collega ogni coppia di vertici con un \textbf{collegamento diretto}. Sulla diagonale ci sono solo 0 (la distanza minima tra un nodo e se stesso è ovviamente 0); se una coppia non è collegata, viene inserito $\infty$.

$$D_{i, j}  [0] = \begin{cases}
0 & se\ i = j \\
w_{ij} & se\ i \neq j \land (i, j) \in E \\
\infty & altrimenti
\end{cases}$$

Questa procedura viene estesa con la possibilità di attraversare un vertice arbitrario, per esempio 1. Se ciò è conveniente, le entrate corrispondenti nella matrice vengono modificate con il nuovo peso minimo ottenuto.

Alcuni percorsi intermedi ottimali possono essere riutilizzati per le caselle successive: se è già stato calcolato un valore minimo per un percorso, quest'ultimo può servire come percorso intermedio tra altri due nodi. 

La domanda è: come costruire la matrice $D(k)$ a partire dalla matrice precedente $D(k-1)$? Si ha:

$$D_{i, j}  [k] = min \begin{cases}
d_{i, j}[K-1] \\
d_{i, k}[K-1] + d_{k, j}[K+1]
\end{cases}$$

La costruzione di tutte le matrici avviene in $\Theta(n^3)$, cioè il tempo di costruzione di una singola matrice moltiplicato per il tempo del confronto, che viene effettuato $(n + 1)$ volte. 

Lo spazio utilizzato è $\Theta(n^3)$, che può essere ottimizzato a $\Theta(n^2)$ considerando che una matrice una volta usata viene eliminata. Se il grafo non è orientato, tutte le matrici sono simmetriche, il che permette di ridurre ulteriormente lo spazio. 

Esiste anche la versione esistenziale di questo algoritmo, i cui valori sono rappresentati da 0 e 1 (\textit{true} e \textit{false}). Al posto del minimo viene usato un $or$ logico, dato che è sufficiente garantire l'esistenza del cammino minimo.

\subsection{Algoritmo}

\begin{algorithm}[H]
    \caption{Floyd-Warshall}
    \begin{algorithmic}
        \Function{FL-WA}{}
            \For{$i \gets 1$ \textbf{to} $n$}
                \For{$j \gets 1$ \textbf{to} $n$}
                    \If{$(i,j) \notin E$}
                        \State $d_{i,j} \gets \infty$
                    \Else
                        \If{$i == j$}
                            \State $d_{0,i,j} \gets 0$
                        \Else
                            \State $d_{0,i,j} \gets \omega_{i,j}$
                        \EndIf
                    \EndIf
                \EndFor
            \EndFor

            \For{$k \gets 1$ \textbf{to} $n$}
                \For{$i \gets 1$ \textbf{to} $n$}
                    \For{$k \gets 1$ \textbf{to} $n$}
                        \State $d_{k,i,j} = \Call{min}{d_{k-1,i,j}, d_{k-1,i,k} + d_{k - 1,k,j}}$
                    \EndFor
                \EndFor
            \EndFor
            \State \Return $D^{(n)}$
        \EndFunction
    \end{algorithmic}
\end{algorithm}

\subsubsection{Esercizio}

\begin{example}{}{}
    Dato un grafo $G = (V, E, W), col: V \rightarrow C$:

    \textbf{Problema}: \textit{Calcolare $\forall (i, j) \in V^2$ il peso di un cammino minimo da $i$ a $j$ che contiene al massimo $N = 2$ coppie di vertici consecutivi di ugual colore.}

    \textbf{Sottoproblema}: \textit{calcolare $\forall (i, j) \in V^2$ il peso del camminimo minimo da $i$ a $j$ che contiene al massimo $m$ coppie di vertici consecutivi di ugual colore, con vertici intermedi $\in \{1, \dots, k\}$}

    Si introduce la variabile associata al problema:

    $D^{k,m}$ i cui elementi sono $\forall (i, j) \in V^2$

    $d_{i,j}^{k, m}$ è il peso di un cammino da $i$ a $j$ \textit{che contiene al massimo $m$ coppie di vertici consecutivi di ugual colore, con vertici intermedi $\in \{1, \dots, k\}$} \\

    \textbf{Caso Base}: $(k, m)$ con $k = 0 \rightarrow \{1, \dots, k\} = \emptyset$

    \begin{align*}
    d_{i,j}^{(k, 0)} &= \begin{cases}
        0 & \text{se }i = j \\
        w_{i,j} & \text{se } i \neq j \land (i, j) \in E \land (col(i) \neq col(j)) \\
        \infty & \text{altrimenti}
    \end{cases} \\
    d_{i,j}^{(k, 1)} &= \begin{cases}
        0 & \text{se }i = j \\
        w_{i,j} & \text{se } i \neq j\\
        \infty & \text{altrimenti}
    \end{cases} \\
    d_{i,j}^{(k, 2)} &= \begin{cases}
        0 & \text{se }i = j \\
        w_{i,j} & \text{se } i \neq j \land (i, j) \in E \\
        \infty & \text{altrimenti}
    \end{cases}
    \end{align*}

    \textbf{Passo Ricorsivo}: $(k, m)$ con $k > 0$ e qualunque $k \in \{0, 1, 2\}$.

    \textbf{Sottocaso}: $k \notin $ cammino
    
    $d_{i, k}^{k, m} = d_{i, j}^{k-1, m}$

    \begin{center}
        \begin{tikzpicture}
            \graph[no placement,nodes={draw,circle},edges={>=latex}] {
                i [at={(0,0)}] -> [decoration={coil}]
                j [at={(2,0)}] 
            };
        \end{tikzpicture}
    \end{center}
    
    \textbf{Sottocaso}: $k \in$ cammino

    \begin{center}
        \begin{tikzpicture}
            \graph[no placement,nodes={draw,circle},edges={>=latex}] {
                i [at={(0,0)}]
                -> [snake=snake]
                k [at={(1,1)}]
                -> [snake=snake]
                j [at={(2,0)}]
            };
        \end{tikzpicture}
    \end{center}
\end{example}