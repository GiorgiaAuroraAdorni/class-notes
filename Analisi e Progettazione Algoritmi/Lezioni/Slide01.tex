\newpage
\section{Inversioni}

\subsection{Il problema}

Come si misura la somiglianza o differenza di gusti? \\
Un esempio banale è il seguente:
\begin{gather*}
    L_A = \{1, 2, 3, 4, 5\}\\
    L_B = \{2, 4, 1, 3, 5\}
\end{gather*} \\
$L_A, L_B$ rappresentano due liste di film ordinati(con gli stessi numeri) e numerati secondo le preferenze rispettivamente di $A$ e $B$. \\
La differenza deve essere $0$ se $L_B = L_A$, e crescere fino ad avere il valore massimo quando $L_B$ è completamente "rovesciata" rispetto ad $L_A$.

\subsection{Il modello}

Una misura naturale è \textbf{il numero di inversioni(o scambi) tra film che devo fare per ricostruire } $L_A$ \textbf{ da } $L_B$ \\
Più in generale, si può arrivare alla seguente definizione:

\begin{definition}[Inversione]
    Data una sequenza di numeri $a_1,a_2,\dots,a_n$, una inversione è una coppia di indici $i < j$ tale che $a_i > a_j$.
\end{definition} ~\\
Questa definizione ci dice praticamente quanti scambi bisogna effettuare affinché l'array risulti ordinato.
$$\begin{tikzpicture}
    \matrix (m) [matrix of math nodes, row sep=2em,
      column sep=1em]{
      L_A & & 1 & 2 & 3 & 4 & 5 \\
      L_B & & 2 & 4 & 1 & 3 & 5 \\};
    \path[-stealth]
        (m-1-3) edge (m-2-5)
        (m-1-4) edge (m-2-3)
        (m-1-5) edge (m-2-6)
        (m-1-6) edge (m-2-4)
        (m-1-7) edge (m-2-7);
\end{tikzpicture}$$

\subsection{Soluzioni}
~
\begin{algorithm}
    \caption{Iterativo}
    \label{inv_it}
    \begin{algorithmic}
        \Function{inverse\_count\_iterative}{$L, n$}
        \State $inv\_count \gets 0$
        \For{$i \gets 0$ \textbf{to} $n$ \textbf{step} $1$}
            \For{$j \gets 0$ \textbf{to} $i + 1$ \textbf{step} $1$}
            \State $inv\_count \gets inv\_count + 1$
            \EndFor
        \EndFor
        \EndFunction
    \end{algorithmic}
\end{algorithm}

Per ogni elemento, si conta quanti elementi alla sua destra sono più piccoli di esso.
\begin{algorithm}
    \caption{Enhanced Merge Sort}
    \label{inv_me}
    \begin{algorithmic}
        \Function{sort\_count}{$L, n$}
            \State $inv\_count \gets 0$
        \EndFunction
    \end{algorithmic}
\end{algorithm}

\newpage
\section{Problema della catena di montaggio}
In una fabbrica di automobili ci sono due catene di montaggio, ciascuna con $n$ stazioni. Due stazioni nella stessa posizione effettuano la stessa operazione ma con tempi differenti, e anche i tempi di entrata e uscita dalle catene possono essere diversi. Il passaggio tra una stazione e un'altra comporta un tempo variabile. \par

Ogni scelta delle stazioni determina univocamente un percorso. Il problema è individuare il percorso che richiede il minimo tempo totale. \\
Calcolare tutte le soluzioni possibili sarebbe computazionalmente impossibile, in quanto richiede un tempo di $2^n$. \par 

Una soluzione ottima si ottiene combinando soluzioni ottime di sotto-problemi. Siano $S_{k_1,\:1},\:S_{k_2,\:2},\:\dots \:,\:S_{k_n,\:n}$ le stazioni relative a una soluzione ottima, con $k_j\:=\:0$ oppure $k_j\:=\:1$. Per ogni $j\:=1,\: \dots, \:n$ la sequenza $S_{k_1,\:1},\:S_{k_2,\:2},\:\dots \:,\:S_{k_j,\:j}$ è la via più breve per arrivare alla stazione $S_{k_j,\:j}$. \par 
Il prossimo passo è esprimere ricorsivamente la soluzione ottima in termine di soluzione di sotto-problemi. 
