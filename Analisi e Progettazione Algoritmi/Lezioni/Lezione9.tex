\newpage
\section{Algoritmi su grafi}
Gli algoritmi su grafi sono rilevanti perché possono risolvere numerose tipologie di problemi, riducendoli appunto a un grafo.

Si ricorda che un grafo è un insieme $G = (V, E)$ dove $V$ sono i vertici, $E$ sono gli archi (edges, lati), ed esiste una funzione peso $w : E \rightarrow \mathcal{R}$. Un cappio viene rappresentato come $e(i,\ i)$. 

Ci sono due modi principali per rappresentare un grafo: 
\begin{enumerate}
	\item Liste di adiacenza, array statico con liste dinamiche, utilizzate per grafi sparsi (con pochi lati);
	\item Matrice di adiacenza, statica, con ogni vertice rappresentato in una rispettiva riga e colonna (per grafi densi).
\end{enumerate}
Per decidere quale delle due usare bisogna confrontare $\abs{E}$ con $\abs{V}^2$: se $E$ è molto minore si procede con la lista, altrimenti la matrice.

\subsection{Liste di adiacenza}
Si ha un array statico $Adj$ di nodi, con un valore numerico univoco assegnato a essi. Ogni vertice è collegato agli altri tramite puntatori: se il grafo non è orientato, può esserci o non esserci il collegamento dall'altro lato (tranne nel caso dei cappi).

Le liste di adiacenza permettono di conoscere quali nodi sono collegati al nodo di partenza, ma non necessariamente ci sono cammini tra più di due di essi. I puntatori $next$, però, sono uno dopo l'altro e ogni singolo vertice ha un solo puntatore (che successivamente punta agli altri).

Tempo peggiore: $O(\abs{V})$, tempo medio: $\Theta(\abs{V}/2)$, tempo migliore: $\Omega(1)$. \\
Spazio: $O(\abs{V} + \abs{E})$ se il grafo è orientato, $O(\abs{V} + 2\abs{E})$ altrimenti.

Se il grafo è pesato, si aggiunge a ogni valore un'altra casella (due campi, quindi), per rendere veloce la ricerca rimettendoci in termine di spazio. 

\subsection{Matrici di adiacenza}
Si ha che la matrice di adiacenza, in caso di grafo non orientato, è simmetrica: quindi non è necessario tenere tutta la matrice in memoria. La rappresentazione e l'accesso in questo caso sono più complicate, quindi per convenzione non si divide.

Per indicare il peso è sufficiente mettere nell'incrocio riga-colonna corrispondente il valore, anziché 1; viceversa possono esserci valori NIL oppure (meglio) 0 o -1.

I tempi sono costanti: per accedere alla matrice è sufficiente trovare la casella desiderata, quindi $\Theta(1)$ nonostante lo spazio sia maggiore, $\Theta(\abs{V}^2)$. Questa soluzione è comunque vantaggiosa se il grafo è denso, perché non è necessario scorrere l'array.

\newpage
\section{Visite di un grafo}
Ci sono due visite principali: BFS e DFS, rispettivamente in ampiezza e in profondità. 

La scelta cambia a seconda di quello che si vuole ottenere: BFS analizza una sola componente del grafo che è connessa alla sorgente, dando informazioni sulle distanze da tutti i vertici (per trovare quella minima), mentre la DFS analizza tutte le componenti dando informazioni su quali vertici sono collegati e quali no.

\subsection{BFS}
La BFS si concentra sui vertici adiacenti, mentre vede come infinitamente distanti quelli non direttamente raggiungibili. Per ogni vertice, si ha il valore della distanza minima (intero o infinito). \\ BFS permette di trovare il \textbf{cammino minimo}. 

Un vertice a distanza $d+1$ viene visitato solo dopo tutti quelli a distanza $d$. Un'altra informazione che di solito viene memorizzata è $p(v)$, cioè il nodo precedente nella visita.

Per capire se un vertice è già stato visistato si usa un colore: bianco se non è mai stato toccato, grigio se è stato incontrato ma non visitato in profondità e nero altrimenti.

L'algoritmo inizia segnando tutte le distanze tra ogni coppia di vertice come infinita, ogni parent è NIL e tutti i colori sono bianchi. Partendo dalla sorgente, li visita tutti da sinistra.

Per capire come trovare i vertici, l'algoritmo deve memorizzare gli $n$ nodi da analizzare successivamente: la struttura dati più comoda è la coda, che garantisce la visita a partire dal primo livello.

Se ci sono due o più grafi scollegati, i nodi vengono inizializzati, ma al termine dell'algoritmo essi restano con distanza infinita: BISF può anche indicare se un grafo è connesso. Se il grafo non è orientato, ogni nodo viene visitato due volte.

% algritmo

\begin{algorithm}[H]
	\caption{BFS}
	\begin{algorithmic}
		\Function{BFS}{G, S}
			\For{$i \gets 1 \textbf{ to } n$}
				\State $d[v_i] \gets \infty$
				\State $c[v_i] \gets $ White
				\State $p[v_i] \gets $ nil
			\EndFor
			\State $d[S] \gets 0$
			\State $c[S] \gets $ Gray
			\State \Call{Enqueue}{Q, S}
			\While{$!(\Call{EmptyQueue}{Q})$}
				\State $u \gets $ \Call{Dequeue}{Q}
				\For{each $w \in $ Adj$[u]$}
					\If{$c[w] == $ White}
						\State $d[w] \gets d[u] + 1$
						\State $p[w] \gets u$
						\State $c[w] \gets $ Grey
						\State \Call{Enqueue}{Q, W}
					\EndIf
				\EndFor
				\State $c[u] = $ Black
			\EndWhile
		\EndFunction
	\end{algorithmic}
\end{algorithm}

\begin{example}{}{}
	Dato un grafo $G = (V, E)$, con $x \in V$, stampare i vertici $y$ che hanno la seguente propretà: il numero di archi su un cammino minimo da $x$ a $y$ è pari.

	\begin{algorithm}[H]
		\caption{Print-Even-Distance}
		\begin{algorithmic}
			\Function{Print-Even-Distance}{G, x}
				\State \Call{BFS}{G, x}
				\For{$n \in V$}
					\If{$d[n] \neq \infty $ and $d[n]$ mod $2 == 0$}
						\State \Call{Print}{n}
					\EndIf
				\EndFor
			\EndFunction
		\end{algorithmic}
	\end{algorithm}
	
\end{example}

\begin{example}{}{}
	Modificare l'algoritmo di visita BFS in modo che visiti fino a scoprire i vertici che distano $k$ da $s$, con $k > 0$ (dentro la padella). 

	\begin{algorithm}[H]
		\caption{BFS}
		\begin{algorithmic}
			\Function{BFS}{G, S}
				\For{$i \gets 1 \textbf{ to } n$}
					\State $d[v_i] \gets \infty$
					\State $c[v_i] \gets $ White
					\State $p[v_i] \gets $ nil
				\EndFor
				\State $d[S] \gets 0$
				\State $C[S] \gets $ Grey
				\State \Call{Enqueue}{Q, S}
				\While{$!(\Call{EmptyQueue}{Q})$}
					\State $u \gets $ \Call{Dequeue}{Q}
					\For{each $w \in $ Adj$[u]$}
						\If{$c[w] == $ White}
							\State $d[w] \gets d[u] + 1$
							\State $p[w] \gets u$
							\If{$d[w] < k$}
								\State $c[w] \gets $ Grey
								\State \Call{Enqueue}{Q, w}
							\Else
								\State $c[w] \gets $ Black
							\EndIf
						\EndIf
					\EndFor
					\State $c[u] = $ Black
				\EndWhile
			\EndFunction
		\end{algorithmic}
	\end{algorithm}
\end{example}

\begin{example}{}{}

	Dato un grafo non orientato $G = (V, E)$, stabilire (true, false) se $G$ è connesso (per ogni coppia di vertici esiste un cammino, relazione di raggiungibilità simmetrica).
\begin{algorithm}[H]
	\caption{Is-Connected}
	\begin{algorithmic}
		\Function{Is-Connected}{G}
			\State S $\gets$ \Call{Random}{V}
			\State \Call{BFS}{G, S}
			\For{$n \in V$}
				\If{$c[n] == $ White}
					\State \Return false
				\Else
					\State \Return true
				\EndIf
			\EndFor
		\EndFunction
	\end{algorithmic}
\end{algorithm}
\end{example}

\begin{example}{}{}
Dato un grafo non orientato $G = (V, E)$, stabilire se $G$ è un albero (grafo connesso senza cicli, oppure $|E| = |V| - 1$). 

Per risolvere il problema è possibile contare il numero di vertici e quello degli archi con variabili globali, e verificare se rispetta la relazione.
\end{example}

\begin{example}{}{}
Dato un grafo non orientato $G = (V, E)$ e un nodo $s \in V$, stabilire se $G$ è un albero con radice in $s$.

Il colore del padre del $u$ è nero. I figli non possono mai essere nella coda insieme al padre, quindi se un vertice è destinato a diventare nero, il padre è necessariamente nero (eheh).

\end{example}