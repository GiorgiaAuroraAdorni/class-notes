\section{Introduzione}

L'analisi dei dati è una disciplina introdotta nel 1935 da \textit{Fisher} (F-test, per capire se due campioni sono originari della stessa popolazione) e in seguito sfruttata per automazione e gestione di dati prevalentemente non strutturati.

Le prime analisi descrittive risalgono alla fine degli anni '70, insieme ai primi software e al linguaggio R, il quale è successivamente stato integrato con tecnologie big data. Le prime modalità di visualizzazione (grafici a torta, istogrammi) nascono per dare una forma ai fenomeni legati al business (\textbf{business intelligence}). 

Alla fine degli anni '90 questi modelli convergono nel \textbf{machine learning}, che permette non solo la creazione di modelli descrittivi ma anche previsioni e prescrizioni in base al contesto. 

Negli ultimi decenni Google ha sviluppato strumenti come TensorFlow per poi renderli accessibili al pubblico, e sono nate diverse discipline e nuove applicazioni di data analytics (Google Flu Trends).

Alcuni utilizzi di questo campo sono i database transazionali per i sistemi di raccomandazione, IoT tramite wireless sensor data, o le analisi in ambito medico. Il volume di dati da elaborare è immensa, essendo questi prodotti in tempo reale dagli utenti e dai dispositivi. 

Per \textbf{data analytics} si intende la generazione di valore da dati per scopi decisionali, cioè la trasformazione dei dati in prodotti. L'intervallo di tempo considerato include presente, passato e futuro, ed è necessaria una profonda comprensione empirica dei modelli.

I big data hanno la caratteristica di essere molto economici e in grande quantità, quindi hanno bisogno di velocità e disponibilità, con query ottimizzate. I sistemi di gestione sono prevalentemente NoSQL. 

Più il dato è complesso, più crescono velocità e volume a discapito della varietà (e viceversa).
\begin{itemize}
	\item Dati strutturati: hanno forma \textbf{proposizionale} (tabellare), permettono query con range numerici e matching esatto di stringhe;
	\item Dati non strutturati: \textbf{grafi} (reti) e \textbf{testo} libero, i quali compongono circa l'80\% del totale dei dati disponibili in un'azienda;
	\item Dati semi-strutturati: via di mezzo tra strutturati e non. La maggior parte dei dati è libero, ma esistono linee guida per le rappresentazioni (ad esempio HTML).
\end{itemize}

