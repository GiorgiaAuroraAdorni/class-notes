\subsection{Wrappers}
I wrappers sono metodologie per riconoscere le possibili interazioni tra le variabili, a discapito del tempo computazionale (al contrario dei filtri).

La subset evaluation viene effettuata tramite wrappers, i quali includono politiche per valutare le features rispetto al potere predittivo del modello di appendimento. Le performances devono soddisfare un determinato livello di qualità per poter accettare il sottoinsieme, e l'obiettivo è la massimizzazione della qualità.

I risultati possono variare a seconda della ricerca nello spazio delle possibili variabili, e l'output è strettamente dipendente dall'approccio utilizzato (paradigma di predizione). Questi parametri devono essere definiti e valutati attentamente, misurando man mano le performance e la robustezza del modello.

Fasi dell'algoritmo:
\begin{enumerate}
	\item Selezione di un subset di attributi (uno o più);
	\item Induzione di un algoritmo di apprendimento (albero di decisione, SVM, \dots);
	\item Valutazione dell'accuratezza o del risultato in generale;
	\item Se essa non è sufficiente, ripetizione del processo.
\end{enumerate}

Il problema della selezione del subset ottimale è NP-hard, quindi si ricorre a strategie metaeuristiche di ricerca. Le classi si dividono in forward e backward (aggiunta o rimozione di features nell'insieme), e la metodologia viene misurata tramite cross-evaluation, inducendo il wrapper sul training set e valutando sul test set. 

La tecnica è comunque molto pesante computazionalmente, però permette un utilizzo semplice e universale dei wrappers.

\section{Evaluation issues}
Lo sviluppo di modelli per la valutazione delle features ha comportato la necessità di una misurazione accurata della predizione, introducendo errori e stime di performances.

Gli errori sul training data non sono buoni indicatori della qualità dell'apprendimento, perché non è possibile prevedere l'output con altri dati (la varianza è alta). I test set per questo sono utili a capire la variabilità dell'output con differenti input.

Problemi più comuni:
\begin{itemize}
	\item Overfitting, quando il modello è molto accurato (troppo complesso) sui dati di training ma ha scarsi risultati con il testing e incapacità di generalizzazione;
	\item Underfitting, quando la varianza è nulla e tutti i dati sono interpretati allo stesso modo troppo generico (semplice), i risultati sono scarsi sia nel training che nel testing.
\end{itemize}

Nel momento in cui i risultati del training iniziano a divergere da quelli del testing, c'è la probabilità di overfitting. Il range ideale ha complessità ridotta del modello con poco scostamento delle curve.

Ci sono diversi metodi di valutazione degli errori:
\begin{itemize}
	\item Misurazione delle performance, contando quante istanze sono classificate correttamente e in base a ciò calcolare error rate e accuracy;
	\item Falsi negativi/positivi, da minimizzare;
	\item Problemi multi-classe, osservando come il modello si comporta in base a ogni classe;
	\item Precisione, tenendo conto delle singole istanze con recall e self-measure;
	\item Curve ROC, grafici di veri positivi e falsi negativi al variare della soglia;
	\item Complessità computazionale.
\end{itemize}

Ognuno di questi è distinto dall'utilizzo di veri/falsi positivi e negativi, che contribuiscono al calcolo di alcuni indicatori di performance. Le misurazioni distinguono tra actual target values $a_1, a_2, \dots, a_n$ e predicted target values $p_1, p_2, \dots, p_n$.
$$\text{Mean-squared error: } \frac{(p_1 - a_1)^2 + \dots + (p_n - a_n)^2}{n}$$
$$\text{Mean-absolute error (meno sensibile agli outliers): } \frac{|p_1 - a_1|^2 + \dots + |p_n - a_n|^2}{n}$$
$$\text{Root ean-squared error: } \sqrt{\frac{(p_1 - a_1)^2 + \dots + (p_n - a_n)^2}{n}}$$

Altre formule utilizzate:
\begin{enumerate}
	\item Accuracy, $A = \frac{\text{numero di istanze correttamente classificate}}{\text{istanze totali}} = \frac{TP + TN}{TP + TN + FP + FN}$;
	\item Precision, $P(k) = \frac{\text{numero di istanze correttamente classificate come }k}{\text{istanze totali classificate come }k} = \frac{TP + TN}{TP + FP}$;
	\item Recall, $R(k) = \frac{\text{numero di istanze correttamente classificate come }k}{\text{istanze totali della classe come }k} = \frac{TP + TN}{TP + FN}$;
	\item F-measure, $F(k) = \frac{2 \cdot P(k) \cdot R(k)}{P(k) + R(k)}$, distinta da:
	\begin{itemize}
		\item Macro-average, $\frac{1}{N} \sum_{i=1}^{N} P(i)$, che consiste nella media di tutte le F-measures di ogni classe;
		\item Micro-average, $\sum_{i=1}^{N} \frac{C_i}{N} P(i)$, per assegnare ad alcune classi una maggiore importanza (dove $C$ è la cardinalità).
	\end{itemize}
\end{enumerate}

\subsection{Numerosità del dataset}
Generalmente il dataset, se sufficientemente grande, viene diviso in training e testing (solitamente 2/3 training, 1/3 testing) ed essi vengono utilizzati rispettivamente per costruire e valutare il classificatore.

Più aumenta di dimensioni il training set, più migliora il classificatore; più aumenta il testing set, più la stima degli errori è accurata. Dopo la valutazione, i dati vengono uniti per costruire la versione finale.

In mancanza di informazioni a riguardo, si assume che la distribuzione dei parametri sia uniforme, e poi si aggiusta la precisione in base alla performance.

Se il quantitativo di dati è piccolo, si usano strategie di repeated holdout: il processo di training viene ripetuto più volte con diversi sottoinsiemi di attributi, ma ciò causa overlap. A ogni istanza viene assegnato un numero casuale per l'assegnazione.

La cross-validation evita l'overlap dividendo i dati in $k$ subsets (di solito 10) e dividendo ulteriormente essi in testing e training (\textit{k}-fold), stratificando i gruppi e calcolando stime globali. La stratificazione permette il rispetto della cardinalità di ogni classe distribuendole comunque in modo relativamente uniforme. 

Se un'istanza di minoranza viene valutata più volte in testing, la sua originale distribuzione marginale viene persa, e ci sono più previsioni per lo stesso dato.

Altre tecniche meno utilizzate sono quelle di bootstrap, con dati poco numerosi: ogni istanza uò far parte sia del set di training che di testing. Ogni elemento viene scelto più volte, il che è utile in presenza di classi di minoranza. La varianza è ridotta, ma i risultati tendono a essere pessimisti.

Una particolare istanza ha probabilità $1 - \frac{1}{n}$ di non essere selezionata, quindi la probabilità di essere selezionata è:
$$\Big(1 - \frac{1}{n}\Big)^n \approx e^{-1} = 0.368$$
Questo significa che il training data ha circa il 63.2\% delle istanze.

Le curve di learning mostrano come l'accuratezza cambia al variare della dimensione del campione, stabilizzandosi dopo una certa quantità. 

Quando ci sono davvero poche istanze si utilizza leave-one-out, una forma di cross-validation che crea tanti folds quante le istanze e ne usa uno solo come testing, iterativamente.









