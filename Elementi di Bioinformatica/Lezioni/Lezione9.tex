\section{Distanza di edit}
La distanza di edit è la trasformazione di una stringa in un'altra tramite inserimento di un nuovo carattere, cancellazione o modifica.

Esempio: $s_1 = ABRACADABRA$, $s_2 = ABRABANANA$, l'obiettivo è trovare il minimo numero di operazioni che trasformano $s_1$ in $s_2$. 

In questo caso, al contrario rispetto all'allineamento, la funzione obiettivo riguarda la minimizzazione: questo è un classico esempio di problema con distanza, utilizzato negli algoritmi di clusterizzazione. 

Il problema è semplice da definire ma non da calcolare, quindi è necessario ricorrere alla programmazione dinamica. 

\subsection{Equazione di ricorrenza}
L'equazione di ricorrenza è simile a quella dell'allineamento. La soluzione ottima è contenuta in una matrice: $M[i, j]$ rappresenta la minima distanza di edit fra i primi $i$ caratteri di $s_1$ e i primi $j$ di $s_2$. 
$$M[i, j] = min \begin{cases}
M[i-1, j-1] & \textrm{se } s_1[i] = s_2[j] \textrm{, cioè nessuna operazione} \\
M[i-1, j] + 1 & \textrm{se viene cancellato un carattere} \\
M[i, j-1] + 1 & \textrm{se viene aggiunto un carattere} \\
M[i-1, j-1] + 1 & \textrm{se viene modificato un carattere}
\end{cases}$$

$$\textrm{Casi limite = }\begin{cases}
	M[0, 0] = 0 \\
	M[i, 0] = 0 \\
	M[0, j] = 0 \\
\end{cases}$$

Per rendere più semplice l'algoritmo, si considera anche l'opzione in cui un'operazione avviene con una coppia di caratteri uguali (che poi non verrà effettuata in pratica, perché sconveniente).

L'ultima componente della soluzione ottima riguarda l'ordine: prima viene modificato il carattere a sinistra, poi quello di destra. 

Considerando anche le no-op, ci saranno caratteri che non vengono mai coinvolti? La risposta è no: al contrario, ogni carattere sarà soggetto a esattamente un'operazione (l'uguaglianza). 

Paragonando questo algoritmo all'allineamento, si ha che l'indel corrisponde a un inserimento o una cancellazione. La distana di edit, quindi, è una versione semplificata della minimizzazione dell'allineamento globale, dove il costo di ogni mismatch è sempre uguale a 1. 

Questo procedimento ha un tempo di $O(nm)$. In bioinformatica, la distanza di edit non è solitamente utilizzata.

\section{Allineamento globale ottimo}
L'allineamento globale richiede tempo e spazio di $O(nm)$. L'occupazione in spazio è più limitante in pratica di quella in tempo: la creazione della matrice è più dispendiosa, ma l'algoritmo funziona sempre qualsiasi siano le stringhe. 

\begin{example}{}{}
	\begin{tabular}{l *{13}{c}}
	$s_1 =$ & A & B & R & A & C & A & D & A & B & R & A \\
	$s_2 =$ & A & B & R & A & B & A & N & A & N & A \\
	$s_3 =$ & A & C & G & T & C & G & G & T & C & C & T & A \\
	$s_4 =$ & C & G & T & G & G & A & A & T & C & A & T & T & T
	\end{tabular}

	L'algoritmo in entrambi i casi ci mette lo stesso tempo perché non è in grado di capire che le prime metà di $s_1$ e $s_2$ sono uguali.
\end{example}

Si vuole considerare l'allineamento globale sotto il vincolo di avere al massimo $k$ colonne con indel o mismatch. 

La programmazione dinamica risolve questo problema con un percorso nella matrice che sarà sempre diagonale tranne al più $k$ volte.

Il percorso contiene al massimo $k$ spostamenti in orizzontale o verticale, questo vale se ci sono $k$ indel. Per accelerare questo processo, è possibile sfruttare il fatto che alcune celle possono non essere calcolate. Il massimo numero di mismatch in una casella $[i, j]$ è $|i-j|$, che dovrà essere strettamente minore di $k$.

Viene considerata solo una banda della matrice, la cui ampiezza sarà al più $2k$. Solo i valori che rientrano in essa vengono calcolati: al massimo ci saranno $2kn$ caselle, mentre il tempo e lo spazio diventano $O(kn)$.

Il problema principale è la decisione del valore di $k$. Il test, fondamentalmente, è: il percorso ottimale con l'allineamento globale è contenuto nella banda? Non ha senso confrontare la matrice ottenuta con quella originaria dell'allineamento, perché troppo dispendioso. 

La stima iniziale più sensata di $k$ corrisponde alla differenza tra le lunghezze delle due stringhe. In seguito, però, bisogna controllare che esso sia ottimale e in caso contrario ridefinire il valore.

Un modo semplice è l'estensione della banda di 1 a destra e 1 a sinistra, calcolare l'allineamento ottimale e verificare che i due siano uguali. In caso contrario, $k$ non è ottimale. Il tempo del controllo è simile.

Quando la stima è sbagliata, la banda può aumentare di ampiezza di una posizione (da entrambi i lati). Il tempo di ciò è $\sum_{i=0}^{k} (2i+n)n = O(k^2n)$, che è troppo. 

Raddoppiando l'ampiezza della banda, $\sum_{i=0}^{\lceil log_2k\rceil} (2^i+1)n \leq \sum_{i=0}^{\lceil log_2k \rceil} 2^{i+1}n = O(kn)$.

