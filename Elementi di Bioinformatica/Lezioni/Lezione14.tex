\section{Approcci basati su parsimonia}
I due problemi confrontati consistono nella grande e piccola parsimonia, risolti con gli algoritmi di \textbf{Fitch} e \textbf{Sankoff}.

Il criterio di parsimonia si basa sul concetto di \textit{rasoio di Occam}, cioè la scelta della soluzione più semplice. L'albero dovrebbe avere la massima verosimiglianza, ma il modello di evoluzione sottostante dev'essere statistico e i tempi di calcolo sono elevati.

\subsection{Grande parsimonia}
Il \textbf{problema della grande parsimonia} consiste nel trovare la filogenesi (l'albero, sconosciuto) che minimizzi il numero di cambiamenti di stato, il momento in cui un carattere viene acquisito. In questo caso la matrice non è binaria, perché ogni carattere può avere più stati, non necessariamente in sequenza.

NB: il problema in versione semplificata considera solo caratteri presenti o meno, quindi la matrice corrispondente è binaria e ogni valore può essere solo 0 o 1.

Questo algoritmo è NP-completo, però può essere ridotto al \textbf{problema della piccola parsimonia}, in cui l'istanza in input è data dalla matrice $M$ e dalla topologia dell'albero. Si vuole determinare quale specie etichetta ogni foglia, e quali passaggi di stato sono rappresentati da ogni arco. 

\subsection{Piccola parsimonia}
La differenza tra la piccola e la grande parsimonia è che nella prima in input c'è la topologia dell'albero, quindi il problema è solo l'etichettatura.

Questo problema, in altre parole, consiste nel riempire l'albero a partire dalla struttura. Esso ha come istanze:
\begin{itemize}
	\item Una matrice triangolare $M$ con $n$ specie e un insieme di caratteri $C$;
	\item Un albero $T$, le cui foglie corrispondono alle specie di $M$;
	\item Per ogni carattere $c \in C$, un costo $w_c$ fra ogni coppia di stati.
\end{itemize}
Una soluzione ammissibile è un'etichettatura $\lambda$ che assegna a ogni nodo uno degli stati possibili per i caratteri $C$. \\
La nostra funzione obiettivo andrà quindi minimizzata:
\begin{equation*}
	min \sum_{c \in C} \sum_{(x, y) \in E(T)} w_c(\lambda_c(x), \lambda_c(y))
\end{equation*}
dove $E(T)$ è l'insieme dei lati di $T$.

\begin{example}{Piccola parsimonia}{}

Si ha in input una matrice $N$ di questo tipo: 
\smallskip
\begin{center}
\begin{tabular}{l | *{6}{c}}
	\# arti & 0 & 1 & 2 & 4 & 6 & 8 \\
	\hline
	0 & 0 & 1000 & 2000 & 4000 & 6000 & 8000 \\
	1 & ~ & 0 & 60 & 150 & 190 & 232 \\
	2 & ~ & ~ & 0 & 30 & 70 & 120 \\
	4 & ~ & ~ & ~ & 0 & 25 & 80\\
	6 & ~ & ~ & ~ & ~ & 0 & 10\\
	8 & ~ & ~ & ~ & ~ & ~ & 0
\end{tabular}
\end{center}
\smallskip
Si indica che il costo di passare da $0$ a $1$ arto è elevato, da $0$ a $2$ arti ancora più elevato, da $6$ a $8$ arti invece è meno elevato, eccetera (in pratica, si assegna un costo legato al numero di arti che la specie aveva prima e dopo essersi evoluta, completamente inventato). È un esempio di caratteri non binari, possono avere più stati. Una matrice di questo tipo è \textit{parte dell'istanza}. \\

\begin{center}
\begin{tabular}{l | c c}
	M & \# arti & Gene X \\
	\hline
	Topo & 4 & 0 \\
	Uomo & 4 & 0 \\
	Ragno & 8 & 1 \\
	Mosca & 6 & 0 \\
	Antilope & 4 & 0
\end{tabular}
\end{center}

\begin{center}
\begin{forest}
	for tree={
		circle, draw, 
		minimum size=1.5em,
		s sep=1cm
	},
	[~, fill=gray,
		[~, fill=red,
			[~, nodevalue={south}{Mosca}],
			[~, nodevalue={south}{Ragno}]
		],
		[~, fill=green,
			[~, nodevalue={south}{Uomo}],
			[~, nodevalue={south}{Topo}],
			[~,nodevalue={south}{Antilope}]
		]
	]
\end{forest}
\end{center}

Nella piccola parsimonia bisogna capire qual è l'etichettatura dei nodi interni (grigio, rosso e verde) in modo che la \textit{somma totale dei costi su ogni arco sia minimizzata}. Si possono trattare separatamente ogni singolo carattere, poiché questi sono \textbf{indipendenti}, quindi l'algoritmo è scritto in maniera tale da trattare un carattere per volta.

\end{example}

\subsection{Algoritmo di Sankoff}
Questo è un algoritmo risolutivo per il problema della piccola parsimonia. Ogni carattere è completamente indipendente, quindi si può lavorare su ognuno singolarmente. 

Per determinare l'etichettatura ottimale, è possibile sfruttare la programmazione dinamica salendo dalle foglie fino alla radice e usando un insieme di sottoistanze ordinate.

L'etichettatura di ogni nodo è l'ultima componente della soluzione nei nodi precedenti. L'ordinamento nell'albero è parziale, ma è sufficiente esaminare le sottoistanze formate dai sottoalberi. Per ogni nodo $x$ nell'albero $T$, il valore $sol(x, y)$ nella matrice sarà la soluzione ottimale del sottoalbero di $T$ che ha radice $X$, con etichetta di $x$ uguale a $y$. 

Formalmente, $sol(x, y) = min_{z_1,\ \dots,\ z_n} \{\sum_{i=1}^{h} sol(f_i, z_i) + d(z_i, y)\}$, dove $\{f_i\}$ sono tutti i figli di $x_iT$ e $\{z_i\}$ sono tutte le possibili etichette di $f_i$.

In altre parole:
\begin{itemize}
	\item $M[x, z]$ è la soluzione ottimale del sottoalbero di $T$ che ha radice $x$, sotto la condizione che $x$ abbia etichetta $z$;
	\item $M[x, z] = 0$ se $x$ è una foglia con etichetta (conosciuta, perché presente nella matrice) $z$;
	\item $M[x, z] = +\infty$ se $x$ è una foglia con etichetta diversa da $z$ (insensata);
	\item $M[x, z] = \sum_{f \in F(x)} min_s \{w(z, s) + M[f, s]\}$ dove $f(x)$ è l'insieme dei figli in $x$ di $T$, se $x$ è un nodo interno;
	\item La soluzione ottimale è $min_s\{M[r, s]\}$ dove $r$ è la radice di $T$.
\end{itemize}

A differenza dell'allineamento, le istanze sono alberi e non stringhe o sequenze; i casi possibili dell'ultimo componente dipendono dal numero di caratteri dell'etichetta, per questo ci si appoggia sull'indice $z$.

Per ogni figlio dell'albero viene calcolata la funzione \textit{min} (tutte le possibilità), e la soluzione del padre viene ricavata da quella dei figli. Ogni cambiamento ha un costo: se i figli sono già etichettati nel modo desiderato allora esso sarà 0.

Il costo dell'arco $(x, f_1)$ è 0 se l'etichetta è uguale a $y$, 1 altrimenti. \\
$z$ rappresenta ogni etichetta diversa da $y$. 

\subsection{Algoritmo di Fitch}
Funziona solo per il caso non pesato, in cui $T$ è un albero binario. Anch'esso sfrutta la programmazione dinamica, utilizzando le seguenti regole:
\begin{enumerate}
	\item $S(x) = \delta_c(x)$ se $x$ è una foglia;
	\item $S(x) = S(f_l) \cap S(f_r)$ dove $f_l$ e $f_r$ sono i figli di $x$ in $T$, se $S(f_l) \cap S(f_r) \neq \emptyset$. \\
	Se l'intersezione non è vuota, si etichetta il padre con l'\textbf{intersezione} dei figli;
	\item $S(x) = S(f_l) \cup S(f_r)$ dove $f_l$ e $f_r$ sono i figli di $x$ in $T$, se $S(f_l) \cap S(f_r) = \emptyset$. \\
	Se l'intersezione è vuota, si etichetta il padre con l'\textbf{unione} dei figli.
\end{enumerate}

La soluzione $B(x)$ consiste nell'insieme degli stati $z$ tale che $M[x, z]$ (il costo del sottoalbero che ha radice $x$ e con stato $z$) sia minimo, con $B(x) = S(x)$. Quindi $B(x)$ è lo stato migliore che posso ottenere con un determinato nodo $x$.
L'algoritmo di Fitch funziona solo con l'albero binario: come estenderlo al caso generico (sempre non pesato)?

Per generalizzare, si usa un'etichettatura con caratteristiche (colori) non pesate, di cui ognuna vale 1, e si cerca l'insieme degli stati che compare il maggior numero di volte.

\section{Filogenesi su distanze}
L'idea di base di questo approccio è che, più piccola è la distanza (non negativa) tra due specie, più simili esse sono.

La distanza è una funzione $S \times S \rightarrow \mathbb{R}^+$, dove $S$ è l'insieme delle specie. Si ricordano le proprietà:
\begin{itemize}
	\item Simmetria;
	\item $d(x, y) = 0 \leftrightarrow x = y$;
	\item Disuguaglianza triangolare.
\end{itemize}

L'input è una matrice (simmetrica) in cui, per ogni coppia di individuo, è contenuta una sua stima della distanza dal punto di vista evolutivo. Ogni arco è etichettato con la misura del tempo trascorso dall'evento di speciazione a oggi. Si otterranno alberi senza radice (questo è un problema).

\begin{example}{}{}
	\begin{center}
	\begin{tabular}{l | *{5}{c}}
		M		& M & U & R & T & A \\ \hline
		Mosca	& - & 10 & 2 & 12 & 9 \\
		Uomo	& ~ & - & 9 & 2 & 3 \\
		Ragno	& ~ & ~ & - & 13 & 14 \\
		Topo	& ~ & ~ & ~ & - & 4 \\
		Antilope & ~ & ~ & ~ & ~ & -
	\end{tabular}

	\begin{forest}
		for tree={circle, draw,minimum size=1.5em, s sep=1cm}
		[~,
			[~, edgelabel={left}{8},
				[~, edgelabel={left}{6}, fill=purple],
				[~, edgelabel={right}{2}, fill=green]
			],
			[~, edgelabel={right}{1},
				[~, edgelabel={left}{3}, fill=gray],
				[~, edgelabel={right}{4}, fill=blue]
			],
			[~, edgelabel={right}{2}, fill=red]
		]
	\end{forest}
	\end{center}
\end{example}

Per ogni arco c'è un intervallo temporale, ed è possibile trovare una matrice delle distanze indotte da $T$ (es. tra il nodo grigio e quello blu c'è una distanza di $4 + 3 = 7$). 

Il problema da risolvere è: data una matrice numerica $M$ di distanze, trovare un albero $T$ la cui matrice di distanza indotta $M_T$ abbia minima distanza da $M$ (sia identica, o abbastanza simile, a quella in input). Non sempre è possibile trovare l'istanza minima, quindi a volte è sufficiente ottenere una matrice abbastanza simile.

Quando una matrice $M$ ha un albero $T$ che la induce?

\subsection{Orologio molecolare}
\begin{example}{}{}

	\begin{center}
	\begin{forest}
		for tree={circle, draw,minimum size=1.5em,s sep=1cm, terminal},
		[~, name=a
			[~, edgelabel={left}{12},
				[~, edgelabel={left}{2}, fill=purple, nodevalue={south}{Mosca}],
				[~, edgelabel={right}{2}, fill=green, nodevalue={south}{Ragno}]
			],
			[~, edgelabel={right}{4},
				[~, edgelabel={left}{10}, fill=red, nodevalue={south}{Uomo}],
				[~, edgelabel={right}{5},
					[~, edgelabel={left}{5},fill=blue, nodevalue={south}{Antilope}],
					[~, edgelabel={left}{5},fill=gray, nodevalue={south}{Topo}, name=b]
				]
			]
		]
		\draw[thick,-latex] ([xshift=1cm]a.north east) -- ([xshift=1cm]b.south east) node[midway, anchor=south,sloped]{tempo};
	\end{forest}
	\end{center}

	La freccia indica che il tempo scorre in egual misura per tutte le specie. Quindi, se il tempo è misurato correttamente, la distanza tra tutti i percorsi da una foglia qualunque alla radice deve essere uguale. Questo deve valere anche per ogni nodo interno: \textit{la distanza fra ogni nodo e tutte le sue foglie deve essere uguale}.
\end{example}

Questo, però, presuppone che si riesca a calcolare una misura corretta del tempo: si possono confrontare solo le sequenze attualmente a disposizione.

L'ipotesi che la differenza genomica sia in funzione proporzionale al tempo trascorso è chiamata \textbf{ipotesi dell'orologio molecolare}, cioè l'ipotesi che il numero di variazioni genomiche sia una misura precisa dello scorrere del tempo.

L'ipotesi non è però vera in maniera assoluta: è troppo semplificativa, quindi è utile solamente per trattare il concetto di distanza. Il caso è ideale, non reale, altrimenti ogni coppia di figli dovrebbe avere la stessa etichetta che li collega al proprio genitore (il che è improbabile). \\
In questo caso, la matrice delle distanze soddisfa la definizione di \textbf{ultrametrica}.

\subsection{Ultrametrica}
Comunque siano prese tre specie, vengono considerate le distanze tra tutte e tre le coppie di punti:
$$\forall\ s_1, s_2, s_3 \quad max\{d(s_1, s_2), d(s_1, s_3), d(s_2, s_3)\}\quad\text{è realizzato da almeno due coppie.}$$

\begin{example}{}{}
	Per esempio, si hanno Uomo($s_1$), Antilope($s_2$) e Topo($s_3$):

	\begin{align*}
		d(s_1, s_2) &= 10 + 5 + 5 &= 20 \\
		d(s_1, s_3) &= 10 + 5 + 5 &= 20 \\
		d(s_2, s_3) &= 5 + 5 &= 10
	\end{align*}

	Due saranno tipicamente più vicini tra di loro ($s_2, s_3$), il resto invece ha valori massimi uguali (uguaglianza triangolare).
\end{example}

Le matrici ultrametriche permettono di usare un algoritmo efficiente per ottenere l'unico albero associato.

Ogni metodo basato su distanze, senza l'ultrametrica che dà una proprietà in più, non è in grado di individuare la radice dell'albero (perché la matrice è simmetrica). Questo è il motivo per cui spesso gli alberi sono senza radice, ed è un problema: la radice serve appunto per capire se le specie hanno dei nodi in comune.

	\begin{figure}[H]
	\caption{Albero senza radice}
	\begin{center}
	\begin{forest}
		for tree={
			circle, draw,minimum size=1.5em,s sep=1cm,
			every leaf node={
				draw,
				rectangle
			}
		},
		[~, fill=blue,rectangle
			[~, 
				[~,
					[~, fill=purple],
					[~,
						[~, fill=red],
						[~, fill=green]
					]
				]
			],
			[~, fill=gray]
		]
	\end{forest}
	\end{center}
	\end{figure}

	\begin{figure}[H]
	\caption{Alberi con radice}
	\begin{center}
	\begin{forest}
		for tree={
			circle, draw,minimum size=1.5em,s sep=1cm,
			every leaf node={
				draw,
				rectangle
			}
		},
		[, phantom, s sep=3cm
			[~,
				[~, fill=purple],
				[~, 
					[~, fill=gray],
					[~, fill=blue]
				],
				[~,
					[~, fill=red],
					[~, fill=green]
				]
			],
			[~,
				[~, fill=blue]
				[~,
					[~,
						[~, fill=red],
						[~, fill=green]
					],
					[~, fill=purple]
				],
				[~, fill=gray]
			]
		]
	\end{forest}
	\end{center}
	\end{figure}

Gli alberi con radice danno delle storie evolutive profondamente diverse: è necessario quindi trovare il modo di capire quale dei nodi interni sia il miglior candidato a radice.

Per poterlo fare, bisogna aggiungere dei cosiddetti \textbf{outgroup}, specie che non c'entrano niente con il soggetto da analizzare.

\subsection{Outgroup}
L'introduzione dell'outgroup serve da una parte per individuare la radice, e dell'altra come \textit{controllo qualità}. È inoltre comune a tutti gli algoritmi che si basano sulle distanze.

A volte si possono introdurre più specie outgroup. Un esempio può essere il seguente:

\begin{example}{}{}
	\begin{center}
	\begin{forest}
		for tree={circle, draw,minimum size=1.5em,s sep=1cm},
		[
			[, fill=black, nodevalue={south}{outgroup}],
			[, name=top
				[, [,name=bottomleft]]
			],
			[, name=topright [], [, name=bottomright]]
		],
		\node[draw=red, thick, fit=(top)(bottomright)(bottomleft)] {};
	\end{forest}
	\end{center}

	La parte evidenziata è la parte di interesse, in cui il nodo padre dell'outgroup viene preso come radice. L'albero evolutivo avrà subito da una parte l'outgroup e dall'altra le specie che interessano.
\end{example}

Se gli outgroup sono nel mezzo e non vicini fra loro e alla radice, la matrice di partenza non era corretta. 


