\documentclass{article}
\usepackage{blindtext}
\usepackage[utf8]{inputenc}
\usepackage{fancyhdr}
\usepackage{amsmath}
\usepackage{amssymb}
\usepackage{amsthm}
\usepackage{tikz}
\usepackage{blkarray}
\usepackage{algorithm}
\usepackage{algpseudocode}
\usepackage{algorithmicx}
\usepackage{hyperref}
\usepackage{tikz}
\usepackage{pgfgantt}
\usepackage{listings}
\usepackage{xcolor}
\usepackage{wrapfig}
\usepackage{relsize}
\usepackage{eurosym}
\usepackage{parskip}
\usepackage{textcomp}
\usepackage{stackengine}
\usepackage{float}
\usepackage{bookmark}
\usepackage{caption}
\usepackage{forest}
\usepackage{mdframed}
\usepackage{cancel}
\usepackage{enumitem}
\usepackage{tabu}
\usepackage{tcolorbox}
\usepackage{multicol}
\tcbuselibrary{theorems,breakable}
\usepackage[a4paper, total={6in, 8in}]{geometry}
\usetikzlibrary{calc,trees,positioning,arrows,chains,shapes.geometric,decorations.pathreplacing,decorations.pathmorphing,shapes,matrix,shapes.symbols,fit,graphs,snakes}
\usepackage{nicefrac}

\title{Elementi di Bioinformatica - Algoritmi}
\author{Adrian Castro, Ilaria Battiston}
\date{Anno scolastico 2018-2019}
\pagestyle{fancy}



% Migliore grafica per i theorem/definition/properties/exercises/examples
\newtheoremstyle{break}% name
{2 \baselineskip}%         Space above, empty = `usual value'
{}%         Space below
{} %{\itshape}% Body font. I prefer empty body font
{\parindent}%         Indent amount (empty = no indent, \parindent = para indent)
{\bfseries}% Thm head font
{.}%        Punctuation after thm head
{\newline}% Space after thm head: \newline = linebreak
{}%         Thm head spec

\theoremstyle{break}

\newtcbtheorem[number within=section]{thm}{Teorema}%
{colback=green!5,colframe=green!35!black,fonttitle=\bfseries}{th}
\newcounter{def}
\newtcbtheorem[number within=section]{definition}{Definizione}%
{colback=red!5,colframe=red!35!black,fonttitle=\bfseries}{dfn}
\newcounter{prop}
\newtheorem{property}[prop]{Proprietà}
\newcounter{exer}
\newtheorem{exercise}[exer]{Esercizio}
\newcounter{exmp}
\newtcbtheorem[number within=section]{example}{Esempio}%
{breakable,colback=gray!5,colframe=gray!35!black,fonttitle=\bfseries}{exmp}
\newcounter{lmma}
\newtheorem{lemma}[lmma]{Lemma}
\tcolorboxenvironment{nota}{
blanker,breakable,left=5mm,
before skip=10pt,after skip=10pt,
borderline west={1mm}{0pt}{red}}

% TikzStyles
\tikzstyle{block} = [rectangle, draw, text centered]
\tikzstyle{empty_block} = [rectangle, text centered]
\tikzstyle{line} = [draw, -latex']
% /End TikzStyles

\definecolor{azzurro}{cmyk}{1,0.33,0,0.13}
\definecolor{arancione}{cmyk}{0,0.41,1,0}
\definecolor{verde}{cmyk}{0.44,0,0.38,0.31}
\definecolor{viola}{rgb}{0.58,0,0.82}
\definecolor{bianco}{rgb}{0.95, 0.95, 0.92}
% C style
\lstdefinestyle{CStyle}{
    backgroundcolor=\color{bianco},   
    commentstyle=\color{verde},
    keywordstyle=\color{viola},
    numberstyle=\tiny\color{arancione},
    stringstyle=\color{azzurro},
    basicstyle=\footnotesize,
    breakatwhitespace=false,         
    breaklines=true,                 
    captionpos=b,                    
    keepspaces=true,                 
    numbers=left,                    
    numbersep=5pt,                  
    showspaces=false,                
    showstringspaces=false,
    showtabs=false,                  
    tabsize=2,
    language=C
}
\lstset{
  basicstyle=\ttfamily,
  columns=fullflexible,
  frame=single,
  breaklines=true,
  postbreak=\mbox{\textcolor{red}{\(\hookrightarrow\)}\space},
}
% maximum matrix environment set to 50 instead of 10
\setcounter{MaxMatrixCols}{50}

\newcommand{\norm}[1]{\left\lVert#1\right\rVert}
\newcommand{\abs}[1]{\left|#1\right|}
\newcommand{\R}{\mathbb{R}}
\newcommand*\circled[1]{\tikz[baseline= (char.base)]{
            \node[shape=circle,draw,inner sep=2pt] (char) {#1};}}
\newcommand{\coloredbox}[1]{\fcolorbox{black}{#1}{\rule{0pt}{6pt}\rule{6pt}{0pt}}\quad}

% Text under-above matrix https://tex.stackexchange.com/a/78467
\newenvironment{spmatrix}[1]
  {\def\mysubscript{#1}\mathop\bgroup\begin{pmatrix}}
  {\end{pmatrix}\egroup_{\textstyle\mathstrut\mysubscript}}

\newenvironment{sbmatrix}[1]
  {\def\mysubscript{#1}\mathop\bgroup\begin{bmatrix}}
  {\end{bmatrix}\egroup_{\textstyle\mathstrut\mysubscript}}

% per le nostre cagate
\newcommand{\checkeditem}{\item[\refstepcounter{enumi}$\text{\rlap{$\checkmark$}}\square$]}
\newlist{todolist}{itemize}{2}
\setlist[todolist]{label=$\square$}

% Euro workaround - https://tex.stackexchange.com/questions/110972/eurosym-seems-to-not-be-working
\DeclareRobustCommand{\officialeuro}{%
  \ifmmode\expandafter\text\fi
  {\fontencoding{U}\fontfamily{eurosym}\selectfont e}}

% forest
\forestset{
  every leaf node/.style={
    if n children=0{#1}{}
  },
  every tree node/.style={
    if n children=0{}{#1}
  },
  edgelabel/.style n args={2}{
    edge label={node[midway,#1]{#2}}
  },
  nodevalue/.style n args={2}{
    label=#1:{{#2}}
  }
}
\forestset{
  suffix tree/.style={
    for tree={
      edge={->},
      every tree node={
        circle, draw, minimum size=1.5em, s sep=1cm,
        s sep+=1em,
        l sep+=1em
      },
      every leaf node={
        rectangle, draw, 
        minimum size=1.5em
      }
    }
  }
}
\forestset{
  highlight ancestors/.style={
    before typesetting nodes={
      for current and ancestors=highlight
    }
  },
  highlight/.style={
    draw=darkgray,
    edge+=red
  }
}
\forestset{
  terminal/.style={
    if n children=0{
    tier=terminal
    }{}
  }
}

\begin{document}

\maketitle

\lfoot{}
\cfoot{}
\rfoot{\thepage}

\newpage
\tableofcontents
\newpage

\section{Classificazione Algoritmi Probabilistici}

Gli algoritmi che trattiamo saranno di due tipi, ed  entrambi prevedono l'utilizzo di un numero pseudo-randomico per il calcolo del risultato:
\begin{itemize}
    \item Monte Carlo: sempre veloci, ma forse non corretti. Il fattore probabilistico è quindi l'errore, che è determinato dalla scelta del numero randomico (es. \textit{Karb Rabin})
    \item Las Vegas: sempre corretti, ma non sempre veloci. La loro velocità dipende dalla scelta del numero randomico iniziale (es. \textit{Quick Sort})
\end{itemize}

\newpage
\section{Pattern Matching}

Di seguito vengono presentati gli algoritmi di pattern matching su stringhe affrontati a lezione.

\subsection{Shift-And}
Algoritmo di string matching che si concentra sul velocizzare il pattern matching effettuando operazioni \textit{bit-a-bit}, mantenendo comunque il numero di operazioni.

\subsubsection{Tempi e spazi}
Il tempo di esecuzione dell'algoritmo è di $O(nm)$, mentre lo spazio occupato è di $\theta (m \times n)$

\subsubsection{Algoritmo}
Definiamo la stringa $T$ di lunghezza $n$ ed il pattern $P$ di lunghezza $m$. Ipotizziamo quindi di avere una matrice $M$ di dimensione $m \times n$ ($m$ righe per le lettere del pattern $P$ e $n$ righe per le lettere della stringa $T$), due indici $i, j$ rispettivamente di $P$ e $T$.

Vale quindi la seguente definizione:
\begin{equation*}
    \texttt{M[i,j] = 1 $\Leftrightarrow$ P[0,i] = T[j-i+1, j]}
\end{equation*}

Tradotto, nella matrice avremo $1$ se e solo se il prefisso di lunghezza $i$ è uguale alla sottostringa di lunghezza $i$ alla posizione $j-i + 1$.

\begin{example}{}{}
    Usiamo $T = AGCAGBGCA, n = 9 \qquad P = GCA, m = 3$
    \begin{center}
    M = \begin{tabular}{l c | *{9}{c} }
        ~ & ~ & 1 & 2 & 3 & 4 & 5 & 6 & 7 & 8 & 9 \\
        ~ & ~ & A & G & C & A & G & B & G & C & A \\
        \hline
        1 & G & 0 & 1 & 0 & 0 & 1 & 0 & 1 & 0 & 0 \\
        2 & C & 0 & 0 & 1 & 0 & 0 & 0 & 0 & 1 & 0 \\
        3 & A & 0 & 0 & 0 & 1 & 0 & 0 & 0 & 0 & 1
    \end{tabular}
    \end{center}
\end{example}
L'ultima riga ci indica se il pattern è presente o meno nel testo, ed in che posizione termina il match.

\subsection{Karp-Rabin}

Si basa sulle moltiplicazioni come operazioni sui bit. Come lo \textbf{Shift-And}, il numero di operazioni rimane invariato, ma si punta a velocizzare le operazioni agendo sui bit.

\subsubsection{Tempi e spazi}

L'algoritmo viene eseguito in un tempo lineare $O(n + m)$, mentre il tempo peggiore è di $O(n \cdot m)$, che accade quando il pattern viene trovato su tutta la stringa.

Per gli hash vengono utilizzati $p^m$, dove $p$ è il numero di simboli presenti nell'alfabeto.

\subsubsection{Algoritmo}

Data una stringa $T$ di lunghezza $m$, un pattern $P$ di lunghezza $n$, e definito l'alfabeto come $\sum = \{0,1\}$, il principio base dell'algoritmo si basa su una \textbf{sliding-window}, di lunghezza $n$, che scorre verso la fine della stringa un carattere per volta, e la funzione \textbf{hash} \texttt{H} che calcola il valore della sliding-window e del pattern che vogliamo riconoscere. Se essi coincidono, allora il pattern è stato trovato alla posizione attuale della sliding-window.

Si riesce a rendere il calcolo dell'hash più veloce a partire dall'hash precedente? Certo che si, usando una tecnica chiamata \textit{rolling hash}:

\begin{equation*}
    \texttt{H(T[i + 1 : i + m]) = (H(T[i : i + m - 1]) - T[i]) / 2 + 2$^m$ $\cdot$ T[i + m]}
\end{equation*}

Tradotto, l'hash della sliding window corrente viene calcolata in base a quella precedente.

In termini di spazio, il primo carattere è quello che pesa di meno ($1$), l'ultimo ha il peso più elevato ($2^{m-1}$) a causa del fattore moltiplicativo. Ad ogni spostamento della sliding-window il primo carattere viene rimosso, il numero viene diviso in due e viene aggiunto l'n-esimo carattere della finestra elevato alla $m$.

Se si vuole cambiare l'alfabeto con $p$ simboli per rendere più veloci le operazioni, bisogna tenere in mente che il peso ne risente ($p^m$), così come l'univocità della stringa associata ad un hash.

\subsection{Pattern Matching con Suffix-Tree}

Questo algoritmo utilizza il concetto dei \textbf{Suffix Tree}, una struttura dati ad albero con archi etichettati. Questa può essere rappresentata come un \textbf{array}, un \textbf{Binary Search Tree} ed una \textbf{tabella hash}.

\subsubsection{Tempi e spazi}

L'algoritmo prima deve costruire l'albero a partire da $T$, impiegando $O(m)$ (se si usa un \textit{array} per immagazzinare l'albero). In seguito, per trovare tutti i match impiega $O(n + k)$, dove $k$ è il numero di occorrenze. Lo spazio occupato è di $O(n^2)$, derivante dalla lunghezza delle etichette. Si riesce ad ottenere approssimativamente $O(n)$ se si utilizzano i puntatori.

\subsubsection{Algoritmo}

Si ragiona definendo la sottostringa da trovare in termini di \textbf{suffisso}. Un suffisso è sicuramente una sottostringa, ed ogni sottostringa è il prefisso di un suffisso. Ogni suffisso corrisponde ad un percorso radice-foglia, di conseguenza si devono cercare i percorsi iniziali.

I passaggi sono i seguenti:
\begin{enumerate}
    \item Per il carattere corrente del pattern, se c'è una corrispondenza con la stringa dell'arco, allora si prosegue verso la direzione dell'arco stesso
    \item Se il pattern non esiste nell'arco corrente, allora esso \textit{non può esistere in tutto l'albero}
    \item Si continua fino ad una o più foglie dell'albero. Il valore delle foglie trovate indicheranno l'\textit{indice al quale si trovano le corrispondenze}.
\end{enumerate}

\subsection{Sottostringa comune di $k$ stringhe}

\section{Minimo tra due intervalli}

\section{Allineamento di sequenze}

\subsection{Allineamento di due sequenze}

\subsection{Allineamento locale}

\subsection{Allineamento globale ottimo}

\section{Gap}

\subsection{Gap arbitrario}

\subsection{Gap lineare}

\section{Matrici di sostituzione}

\section{Allineamento multiplo}
\end{document}