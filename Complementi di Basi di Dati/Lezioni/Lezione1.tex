\section{Introduzione}
I database non consistono solo in dati e tabelle: sono modelli centralizzati o distribuiti che permettono di gestire un carico anche significativo di utenti. La sicurezza è un aspetto importante, insieme all'affidabilità: i sistemi relazionali hanno grandi utilizzi nei settori bancari, e le operazioni devono giungere a termine senza guasti.

I DBMS sono quindi sistemi che devono garantire la gestione di dati di grandi dimensioni, persistenti, affidabili e condivisi. Oltre ai modelli relazionali esistono i NoSQL (Not Only SQL) e RDF, i quali hanno vantaggi e svantaggi a seconda dell'utilizzo.

Per garantire le precedenti caratteristiche, l'architettura di un DBMS deve avere una serie di funzionalità cooperanti, come un gestore delle transazioni, un query compiler e un gestore della memoria secondaria.

\section{Strutture fisiche di accesso}
La più piccola struttura di memorizzazione dei dati a cui gli utenti possano accedere è il file: così come le tabelle, ha un'intestazione fissa e un numero di righe.

I dati non devono solo essere memorizzati in modo persistente: devono anche essere facilmente recuperabili, e gli accessi da gestire sono sia in lettura che in scrittura. Un obiettivo fondamentale è la minnimizzazione del tempo di accesso e trasferimento da CPU alla memoria secondaria.

I dati sono memorizzati nei dischi magnetici, con blocchi da 4-32kbyte e un tempo di accesso di circa $10^{-8}$ millisecondi, con un tasso di trasferimento di 300 Mbit/secondo. Il problema è la grande differenza di ordini di grandezza tra queste operazioni.

L'organizzazione ottimale è un compromesso tra il tempo e lo spazio, di cui il tempo è la priorità considerando il costo ridotto della memoria. 

\subsection{Campi in SQL}
Ogni campo in una tabella è memorizzato in una struttura fisica, ma lo spazio occupato cambia a seconda del tipo del dato. Le tuple sono pertanto record fisici organizzati in collezioni all'interno dei blocchi di memoria, e bisogna gestire anche modifica e cancellazione.

Esempi di tipi di dati:
\begin{itemize}
	\item VARCHAR, che alloca $n + 1$ bytes secondo un bit di carattere separatore o numero di caratteri;
	\item BLOB e GLOB, destinati a larghi file il cui spazio viene allocato solo al momento dell'inserimento.
\end{itemize}

I record possono avere formato e lunghezza fissi o variabili, con eventuali campi straordinari. Il record layout include informazioni supplementari con schemi o puntatori, lunghezza e timestamp di ultima lettura e scrittura. 

I record sono organizzati in blocchi, unità di memoria trasferite dal disco alla memoria principale. La dimensione è generalmente fissa a $2^n$, di cui solitamente alcuni byte non vengono utilizzati. 

Nel caso in cui la lunghezza sia variabile, lo header contiene anche la lunghezza del record e l'offset (distanza rispetto al byte iniziale) dei campi. I campi fissi sono allocati prima di quelli variabili. 

Alcuni record sono rappresentati in più blocchi, cioè spanned: anche questa informazione è contenuta nell'header, e l'ordinamento è gestito tramite la contiguità fisica o collegamento tra record (modello a grafo, sistemi scalabili). 

