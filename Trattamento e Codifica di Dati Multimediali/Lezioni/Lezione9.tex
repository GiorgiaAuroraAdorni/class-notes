\section{Sistemi LTI e trasformata zeta}
La funzione di trasferimento $H(z)$ può essere espressa in termini delle radici dei polinomi al numeratore e al numeratore fattorizzando i due polinomi $N(z)$ e $D(z)$:
% formula slide 30

La frequenza è legata alla posizione angolare, quindi essendo i poli in zero la loro influenza si riflette su tutte le frequenze o nessuna: il contributo è nullo.

Studiando i poli si analizza la regione in cui la funzione viene amplificata e cresce in valore, gli zeri dove diminuisce. Insieme quindi danno l'andamento delle frequenze.

I coefficienti nell'equazione alle differenze corrispondono ai coefficienti nella trasformata, considerando il cambiamento di segno. Al numeratore c'è la parte non ricorsiva, e al denominatore quella ricorsiva. Se non c'è il denominatore, la sequenza è FIR.

Il sistema è stabile se la trasformata di Fourier è interna alla regione di convergenza. 

I sistemi FIR hanno risposta d'impulso limitata, e si può immediatamente passare alla rappresentazione tramite equazione alle differenze dato che i coefficienti sono gli stessi. I sistemi non sono ricorsivi.

\subsection{Stabilità BIBO}
La stabilità BIBO di un sistema LTI impone che la risposta all'impulso $h(n)$ sia sommabile in modulo. Se il sistema è causale, la condizione si traduce nel dominio $z$ imponendo che la funzione di trasferimento $H(z)$ abbia poli contenuti nella circonferenza di raggio unitario del piano $z$. 

Le caratteristiche della sequenza dipendono dalla posizione dei poli: la convergenza, e conseguentemente la stabilità, sono determinate dall'appartenenza al cerchio unitario.

I poli all'esterno del cerchio $\abs{z} = 1$ ...
Poli multipli sul cerchio unitario conducono a una crescita di tipo polinomiale. 

Riassumendo:
\begin{itemize}
	\item Se il sistema è causale, condizione necessaria e sufficiente per la stabilità BIBO è che $H(z)$ abbia tutti i poli contenuti nella circonferenza di raggio unitario;
	\item ...
\end{itemize}

Quando il polinomio è espresso tramite potenze negative, se c'è una costante essa corrisponde all'assenza di ritardo, quindi sia al numeratore che al denominatore si somma 1 per indicare l'istante corrente. 

Tutte le volte che la soluzione non è a coefficienti reali, è presente il complesso coniugato. I monomi vengono espressi in termini di potenze negative di $z$.

\subsection{Filtri}
Avendo a disposizione la trasformata con i poli e gli zeri, conoscendo le proprietà di continuità della funzione analitica è possibile conoscere l'andamento delle frequenze.

Se un polo reale è molto vicino al cerchio di raggio unitario, si ha che il denominatore si annulla con valori prossimi a $\rho = 1$, ma a $z = 1$ c'è un picco e la risposta cresce in funzione alla posizione del polo. In altre parole, ogni polo ha un'influenza sulle frequenze a seconda della loro vicinanza.
 
Introducendo un fattore di normalizzazione $1 - \alpha$, è possibile confrontare i filtri con valore di picco (guadagno) unitario.

Per enfatizzare le basse frequenze, è sufficiente inserire degli zeri per farle crescere di valore. Creando monomi con radici e aggiungendole al numeratore e al denominatore, si creano filtri in grado di alzare o abbassare le frequenze manipolando le posizioni di poli e zeri.

Nella realtà bisogna considerare che è impossibile azzerare completamente una parte di frequenza, ma ci saranno delle oscillazioni più o meno ampie.

Tanto più i poli sono vicini all'asse reale, più influenzano le frequenze basse (filtro passa-basso).

Se la frequenza è causale, il numero di zeri non deve superare il numero di poli.
% foto sistemi
Ricostruendo la frequenza a partire dal filtro, si ha $H(z) = k \frac{z}{denz - \alpha} = \frac{k}{1 - \alpha z^{-1}}$ con $k$ costante. Se il guadagno è unitario, $\abs{H(z)_{f=0 \land z=1} = 1}$ da cui $k = 1 - \alpha$ (filtro con sempre la stessa altezza).

\subsection{Sistema inverso}
Un sistema inverso ha lo scopo di compensare i poli e gli zeri, ed è un sistema che inverte il comportamento di un altro sistema LTI tale che la risposta rimanga inalterata.

Il numeratore di uno dev'essere uguale al denominatore dell'altro e viceversa, e la funzione di trasferimento è pari a una costante unitaria nel piano $z$ (identità).

\subsection{Sistemi passa-tutto}
Questi filtri non modificano le frequenze del segnale, ma effettuano altre operazioni (ritardo, anticipo), e dev'esserci una precisa relazione tra i poli e gli zeri. La risposta all'impulso è sempre costante.

I coefficienti del polinomio al numeratore sono in ordine inverso rispetto a quelli del denominatore, e ciò viene garantito imponendo il modulo quadro unitario. Raccogliendo $z$, il numeratore resta invariato e il denominatore viene invertito. 

Un polo, quindi, è il reciproco di uno zero (e viceversa). Se uno è all'interno della circonferenza, l'altro è fuori.



