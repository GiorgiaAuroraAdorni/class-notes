\section{Analisi di Fourier}
L'analisi di Fourier decompone il segnale in costituenti sinusoidali di differenti frequenze. Il segnale non è più nel dominio tempo-spazio, ma delle frequenze: i dati sono gli stessi, cambia solo la rappresentazione.

\textit{Ogni funzione periodica e a quadrato sommabile può essere espressa come somma di funzioni seno e coseno (combinazioni di funzioni armoniche).}

Si ricorda che una sequenza periodica è $x(n) = x(n + T)$. Una funzione armonica è una funzione periodica del tipo:
$$y = A\sin(\varpi x + \varphi) \qquad y = A\cos(\varpi x + \varphi)$$
Dove $A$ è l'ampiezza, $\varpi$ è la pulsazione, $\varphi$ è la fase. Si ha che $\varpi = 2\pi/T$ dove $1/T$ è la frequenza, e $\pi$ è $180^{\circ}$.