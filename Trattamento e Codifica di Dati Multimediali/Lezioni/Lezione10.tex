\section{Colore e visione}
L'immagine a colore viene percepita dal sistema visivo grazie alle onde elettromagnetiche, caratterizzate dalla lunghezza d'onda (proporzionale all'inverso della frequenza) e dall'ampiezza. La lughezza d'onda è rapportata il periodo: indica quanto ci mette un ciclo a compiersi.

Lo spettro è in buona parte invisibile, con estremi che corrispondono alle radiazioni ultraviolette e infrarossi. La porzione visibile è misurata in nanometri, con ordini di grandezza molto alti.

Il colore è processato nel cervello da coni e bastoncelli, in numerose tipologie con risposte differenti a seconda della lunghezza d'onda. I bastoncelli sono in percentuale estremamente maggiore, ed essendo essi atti a misurare la variazione d'intensità, gli esseri umani hanno più sensibilità in condizioni di scarsa luce. Le due dimensioni sono separabili. 

L'occhio è più sensibile alle variazioni di luce nel centro dello spettro visibile, quindi lo spettro avrà la forma di una curva e la funzione varia al variare delle lunghezze d'onda. A seconda dei nm della radiazione, alcuni coni saranno attivati, eventualmente fondendo le risposte. 

\subsection{Segnale reale}
L'oggetto assorbe una parte della radiazione e ne ritrasmette un'altra parte (parametri legati alla capacità di riflettere), i cui segnali interferiscono con i recettori umani. Ogni colore può essere descritto da tre parametri (RGB).

I numeri in output si definiscono valori di stimolo, che poi si traducono nei canali RGB. Le distribuzioni continue in funzione di $\lambda$ si trasformano in valori discreti in spazio tridimensionale grazie all'integrale.

Le immagine acquisite da una fotocamera digitale sono differenti da quelle elaborate dall'occhio umano: ogni camera ha una matrice che processa il colore in modo da renderlo simile all'output del cervello, e i valori ottenuti dai sensori sono interpolati. 

I dati sono fatti passare in filtri colore, disposti secondo Bayer pattern. Il verde è il colore recepito meglio, e permette di misurare maggiormente le variazioni di intensità (e conseguentemente i dettagli dell'immagine). 

La somma dei contributi è chiamata sintesi additiva; esiste anche la sintesi sottrattiva, che fa differenze in base allo spettro e ai colori complementari di RGB (ciano, magenta, giallo). 

Esistono rappresentazioni che separano i canali in base alla loro intensità, mettendola in evidenza in base alla frequenza delle onde. 

Le immagini hanno due tipologie di formato: \\
Raster, scalate in base a un numero prefissato di pixel in una griglia di elementi; \\
Vettoriale, rappresentate con formule matematiche.
   
\subsection{Istogramma}
Un istogramma rappresenta l'insieme dei valori che un segnale può assumere, espresso in livelli di quantizzazione. Non dà informazioni spaziali, ma solo sul contrasto e sul numero di bit: indica quanti campioni assumono un certo valore (quantizzatore ottimo), e in base a ciò è possibile decidere come applicare la compressione (probabilità normalizzata).

HDR: usa n bit per catturare una porzione del range dinamico con differente esposizione, per poi sovrapporre tutte le immagini ottenute in modo da avere una visione più chiara. 

Per rappresentare meglio immagini con bianco o nero predominante, si utilizzano quantizzatori con pochi bit tagliando tutti i valori che non vengono utilizzati, in modo da avere una distribuzione più uniforme. 

Effettuando il dithering (evitare i salti), bisogna anche aggiungere rumore casuale: in questo modo le basse frequenze vengono nascoste dalle alte, e interferiscono con il rumore di quantizzazione.

\section{Audio}
Un segnale sonoro è la variazione di pressione sul timpano, percepito nel tempo attraverso l'aria (mezzo di propagazione). Il suono è un'onda di pressione come la luce, ma macroscopico.

L'aumento della frequenza non è linearmente in rapporto con l'ampiezza, ma si comporta seguendo una scala logaritmica. Il range di suoni in grado di essere percepite dagli umani è minore dell'insieme delle frequenze possibili, oltre esso ci sono ultrasuoni e infrasuoni. 

Le curve isofone rappresentano i suoni all'interno di una soglia di udibilità, che può essere superata fino al fastidio. 

Il segnale vocale può avere componenti fino a 10 KhZ, ma in genere se ne utilizzano solo 8, quindi questa è la frequenza di campionamento del segnale telefonico. 






