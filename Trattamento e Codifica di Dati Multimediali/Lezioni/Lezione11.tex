\section{Audio}
Un segnale sonoro è la variazione di pressione sul timpano, percepito nel tempo attraverso l'aria (mezzo di propagazione). Il suono è un'onda di pressione come la luce (fenomeno ondulatorio), ma macroscopico: le molecole dell'aria vengono compresse ed espanse, e la sorgente sonora vibra in modo longitudinale nella stessa direzione di propagazione del suono.
$$A(t) = A_{max} \cdot \sin(2\pi ft + \varphi_0)$$
I suoni elementari hanno andamento sinusoidale, periodico e con estensione indefinita. La maggior parte dei suoni natura sono caratterizzati da forme d'onda diverse, ma possono essere scomposte come una combinazione di suoni elementari.

Le componenti di un segnale sonoro sono individuate da:
\begin{itemize}
	\item Ampiezza $A$, che si misura rispetto al valore medio della pressione dell'aria ed è espressa in dB;
	\item Periodo $T$, la durata nel tempo di ogni ciclo dell'oscillazione, espresso in secondi;
	\item Frequenza $F$, velocità con cui i valori di pressione fluttuano ciclicamente, espressa in numero di cicli al secondo (onde e Hz).
\end{itemize}

% immagine onde suono
Il valore 0 indicato sull'asse della pressione corrisponde al valore medio della pressione nell'aria. La differenza di fase ha unicamente a che vedere con il fatto che due funzioni siano diversamente allineate rispetto al tempo.

\subsection{Analisi di Fourier}
L'analisi di Fourier permette la rappresentazione del segnale sonoro nel dominio delle frequenze a partire dal tempo, esplicitando $f$.
% onde nere
La presenza di una linea nello spettro in frequenza indica la presenza di un segnale esattamente sinusoidale periodo, tuttavia i suoni caratterizzati da uno spettro discreto sono pochi. I suoni normalmente uditi hanno un inizio e una fine precisi, cioè sono contenuti in un intervallo temporale finito.
% tempo frequenze

Ai parametri che descrivono un segnale ondulatorio possono essere associate le tre grandezze percettive che descrivono ogni suono:
\begin{itemize}
	\item Altezza, che rappresenta la tonalità dell'audio e ha come parametro la sequenza;
	\item Intensità, il volume, con parametro fisico l'ampiezza;
	\item Timbro, cioè la tipologia di strumento, con parametro fisico lo spettro.
\end{itemize}

A parità di frequenza fondamentale e intensità, due suoni possono differire per timbro (la sovrapposizione delle onde sinusoidali può essere diversa).

La frequenza fondamentale è proporzioanale all'altezza del suono, cioè della sensazione di acutezza/gravità. L'aumento della frequenza non è linearmente in rapporto con l'ampiezza, ma si comporta seguendo una scala di tipo logaritmico. Affinché in un suono sia possibile individuare un'altezza, esso deve essere periodico. 

\subsection{Grandezze fisiche e grandezze percettive}
Ampiezza e frequenza della forma d'onda hanno effetto sul suono percepito: variazioni di piccola ampiezza producono suoni di bassa intensità e viceversa, e al crescere della frequenza aumenta il tono (non linearmente).

Le energie in gioco nei fenomeni acustici sono irrilevanti rispetto a quelle nel fenomeno luminoso. Il range di suoni in grado di essere percepite dagli umani (tra i 20 e i 20.000 Hz) è minore dell'insieme delle frequenze possibili, oltre esso ci sono ultrasuoni e infrasuoni. 

Il volume aumenta man mano che l'ampiezza cresce, e l'incremento di tono è sempre più piccolo al crescere della frequenza.



Le curve isofone rappresentano i suoni all'interno di una soglia di udibilità, che può essere superata fino al fastidio. 

Il segnale vocale può avere componenti fino a 10 KhZ, ma in genere se ne utilizzano solo 8, quindi questa è la frequenza di campionamento del segnale telefonico. 